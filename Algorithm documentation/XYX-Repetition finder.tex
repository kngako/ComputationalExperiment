\documentclass[12pt]{article}

\usepackage[english]{babel}
\usepackage{longtable}
\usepackage{pdfpages}
\usepackage{amsthm}


\newtheorem{defn}{Definition}[section]
\newtheorem{obser}{Observation}[section]
\begin{document}

\begin{center}
	\LARGE\textbf{Repeating sub-string finder algorithm}
\end{center}

\newpage
\tableofcontents
\newpage
\section{Introduction}
This document purposes a viable algorithm for finding all the repetitions of all the repeating sub-strings in an arbitrary string. Repeating sub-strings are defined by this document as follows:\\
\begin{defn}
    Repeating sub-string $Z$.   
	\begin{enumerate}
		\item Let $A$ be an arbitrary selected alphabet.
		\item Let $S$ be an arbitrary selected string that consists of two or more characters that are elements of $A$.
		\item Let  $S^{\prime}$ be a sub-string of $S$.
		\item If there is another sub-string,  $S^{\prime\prime}$, in S that is identical to  $S^{\prime}$ but does not start at the index of  $S^{\prime}$ in S then the sub-string  $S^{\prime}$ is repeated in S and the repeating content of the sub-strings is then denoted by $Z$. The sub-strings $S^{\prime}$, $S^{\prime\prime}$, etc are donated as $Z_{1}$,$Z_{2}$, etc which is generalised to $Z_{i}$ where i indicates the sequential index of the repeating sub-string in relation to all repeating sub-strings of $Z$.
		\item $Z_{1}$ indicates the first occurrence of $Z$ in $S$ from left to right and $Z_{N}$ indicates the last occurrence of $Z$ in S. Where $N$ is defined as the number of occurrences of a particular repeating term.  
	\end{enumerate}
\end{defn}
The document will further propose an extension of the algorithm that will enable it to find the all the X-Y-X repetitions that occur in the arbitrary sting S. A X-Y-X repetition is defined by this document as follows:   
\begin{defn} X-Y-X Repetition
	\begin{enumerate}
		\item Let A be an arbitrary selected alphabet.
		\item Let S be an arbitrary selected string that consists of two or more characters that are elements of A. 
		\item Let X be a non-zero length sub-string of S which repeats M times in S. Where $ M \textgreater 1 $. This is the repeating term of the X-Y-X repetition.		
		\item  Let Y be any of the M - 1 sub-strings of S, which consists of zero or more characters and which starts directly after the end of some occurrence of $X_{i}$ and ends directly before the occurrence of $X_{j}$. Where $j = i + 1$.
		\item Hence a X-Y-X repetition is defined as the M identical occurrences of X in S that are each separated by one of the M - 1 varying sub-strings Y.	
	\end{enumerate}
\end{defn}

The above definitions and terms lay the bases of the algorithm and will be used throughout the document.

\section{Algorithm} 
In order to provide a formal description of the algorithm purposed above the \textit{Repeating sub-string} definition must be used in conjunction with following observation:
\begin{obser} 
Repeating sub-strings are identical to the prefix, of length $|Z|$, of the sub-string of S starting at the starting index of the first occurrence of $Z$, $Z_{1}$, in S. 
	\begin{enumerate}		
		\item Let R be the set of all non-identical repeating sub-string content Z in S.
		\item Let $R_{i}$ be the set of repeated sub-strings ${Z_{1},...,Z_{N}}$ for a particular sub-string content $Z$.
		\item Let $S^{\prime}$ be the sub-string of S which starts at the index of $Z_{1}$ in S of some set $R_{i}$ and extends to last index in S.
		\item Therefore $S^{\prime}$ contains the N occurrences of $Z$ in S. Since $Z{1}$ to $Z_{N}$ all in S and are sequential.
		\item Let P be the prefix of $S^{\prime}$ with a length of $|Z|$ of $R_{i}$.		
		\item Since $Z_{1}$ is the start of $S^{\prime}$. It is also P.
		\item It can be observed that each occurrence of $Z$, $Z_{i}$, is identical to the P.
	\end{enumerate}
\end{obser}

\begin{obser}
Each occurrence of $Z$ in the string S forms a cluster of $\frac{|Z|(2 + (|Z|-1))}{2}$ repeating sub-strings of S.
	\begin{enumerate}
		\item $Z$ has two or more occurrences in $S$ by definition.
		\item Each occurrence, $Z_{i}$, contains one or more characters by definition.
		\item Splitting the occurrence,$Z_{i}$, into J layers denoted by L, where J is $|Z_{i}|$ so that each layer $L_{j}$, where $0 \leq j \textless J $, contains $J - j$ equal length partitions of the sub-string $Z$, where each partition's starting index does not duplicate any other partition's starting index on the same layer.
		\item Each of the partitions on each layer can be seen as a repeating sub-string in S since there is more than one occurrences of it in S since there is more then one occurrence of $Z$ as shown above. 
		\item Since each layer adds J - j to the total partitions for a particular value of J. Hence the number of partitions contained by a sub-string $Z$ is the summation of the layers which is $J + J - 1 + J - 2 + ... + 2 + 1$. This is an arithmetic sequence that can be described by the explicit formula $a_{n} = a_{1} + (n - 1)d$. Thus the summation of these partitions can be described by the explicit formula $\frac{J(2 + (J-1))}{2}$. 
		\item Therefore $Z$ forms $N$ clusters containing $\frac{|Z|(2 + (|Z|-1))}{2}$ repeated sub-strings. Since $J = |Z_{i}| = |Z|$.
		      
	\end{enumerate}	
\end{obser}

The above observations forms the core characteristics of a repeating sub-string and the string that contains it. These characteristics are exploited by the algorithm. The algorithm consists of two stages and a third stage that allows for the extension of the algorithm as stated in the introduction of this document.

\subsection{Stage 1: Finding maximum longest prefixes of each character in S}
\begin{enumerate}


\item The first stage splits the arbitrary string S into a array of strings, $K$, of $|S|$ sub-strings. Where the first sub-string in the array is $S$.
Each of the sub-strings in $K$ have the following properties:
\begin{enumerate}
	\item Is donated by $K_{i}$, where i is its index in the array and $ 0 \le i \textless |S|$.
	\item Has a length $T = |K_{i-1}| - 1$ or $T = |S| - i$, one less the length of the sub-string prior to it in $K$.
	\item Is a sub-string of S which starts at the character $C_{|S| - T} $ in S and ends at character $C_{|S|}$ in S.
\end{enumerate}

\item Next the algorithm creates a numeric array, $U$, which has a length of $|S|$ and all of its elements are initialised to zero. 
U has the following properties:
\begin{enumerate}
	\item Each element, $E_{i}$ is mapped to the corresponding character $C_{i}$ in S, where $ 0 \le i \textless |S|$.
	\item Each element has numeric value $\ge 0$.
	\item Each element represents the maximum longest prefix of all the longest prefixes found in each of the sub-strings in $K$ which have a character index $y$ which can be mapped to the character index $C_{i}$ in S. Where the character index $y$ donates the starting index of the prefix of particular sub-string in $K$. The mapping is defined as $i = y + (|S| - z)$, where $z$ is the index of sub-string in $K$ and $y$ is the character index in $K_{z}$.   
	
\end{enumerate}

\item Next each sub-string, $K_{z}$, in K is then scanned iteratively for the longest prefix sub-string starting at each character, $K_{zy}$. If the value of the element in U which is mapped to by the character index $y$ of the sub-string $K_{z}$, is less than the length of longest prefix starting at $K_{zy}$ then it is replaced by the length of the longest prefix starting at $K_{zy}$. \\
\end{enumerate}
  
In order to find the longest prefixes a modified version of the O(n) longest prefix sub-string finder algorithm created by Michael G. Main and Richard J. Lorentz and documented in their research article \textit{An O(n log n) Algorithm for finding all repetitions in a String*} which was published in the Journal of Algorithms in 1984.\\

The pseudo code of the stage 1 algorithm follows:
\end{document}