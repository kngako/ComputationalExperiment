\documentclass[12pt]{article}

\usepackage[english]{babel}
\usepackage{longtable}
\usepackage{pdfpages}
\usepackage{amsthm}


\newtheorem{defn}{Definition}[section]
\newtheorem{obser}{Observation}[section]
\begin{document}

\begin{center}
	\LARGE\textbf{Repeating sub-string finder algorithm}
\end{center}

\newpage
\tableofcontents
\newpage
\section{Introduction}
This document purposes a viable algorithm for finding all the repetitions of all the repeating sub-strings in an arbitrary string. Repeating sub-strings are defined by this document as follows:\\
\begin{defn}
    Repeating sub-string    
	\begin{enumerate}
		\item Let $A$ be an arbitrary selected alphabet.
		\item Let $S$ be an arbitrary selected string that consists of two or more characters that are elements of $A$.
		\item Let  $S^{\prime}$ be a sub-string of $S$.
		\item If there is another sub-string,  $S^{\prime\prime}$, in S that is identical to  $S^{\prime}$ but does not start at the index of  $S^{\prime}$ in S then the sub-string  $S^{\prime}$ is repeated in S and the repeating content of the sub-strings is then denoted by $Z$. The sub-strings $S^{\prime}$, $S^{\prime\prime}$, etc are donated as $Z_{1}$,$Z_{2}$, etc which is generalised to $Z_{i}$ where i indicates the sequential index of the repeating sub-string in relation to all repeating sub-strings of $Z$.
		\item $Z_{1}$ indicates the first occurrence of $Z$ in $S$ from left to right and $Z_{N}$ indicates the last occurrence of $Z$ in S. Where $N$ is defined as the number of occurrences of a particular repeating term.  
	\end{enumerate}
\end{defn}
The document will further propose an extension of the algorithm that will enable it to find the all the X-Y-X repetitions that occur in the arbitrary sting S. A X-Y-X repetition is defined by this document as follows:   
\begin{defn} X-Y-X Repetition
	\begin{enumerate}
		\item Let A be an arbitrary selected alphabet.
		\item Let S be an arbitrary selected string that consists of two or more characters that are elements of A. 
		\item Let X be a non-zero length sub-string of S which repeats M times in S. Where $ M \textgreater 1 $. This is the repeating term of the X-Y-X repetition.		
		\item  Let Y be any of the M - 1 sub-strings of S, which consists of zero or more characters and which starts directly after the end of some occurrence of $X_{i}$ and ends directly before the occurrence of $X_{j}$. Where $j = i + 1$.
		\item Hence a X-Y-X repetition is defined as the M identical occurrences of X in S that are each separated by one of the M - 1 varying sub-strings Y.	
	\end{enumerate}
\end{defn}

The above definitions and terms lay the bases of the algorithm and will be used throughout the document.

\section{Algorithm} 
In order to provide a formal description of the algorithm purposed above the \textit{Repeating sub-string} definition must be used in conjunction with following observation:
\begin{obser} Repeating sub-strings are identical to the prefix, of equal length, of the sub-string of S starting at the starting index of $Z_{1}$. 
	\begin{enumerate}
		
		\item Let R be the set of all non-identical repeating sub-string content Z in S.
		\item Let $R_{i}$ be the set of repeated sub-strings ${Z_{1},...,Z_{N}}$ for a particular sub-string content $Z$.
		\item Let $S^{\prime}$ be the sub-string of S which starts at the index of $Z_{1}$ in S of some set $R_{i}$ and extends of X in S and extends to the last character in S.
		\item Let P be the prefix of $S^{\prime}$ with a length of $|Z|$.
		\item Therefore M contains the N occurrences of X in S.
		\item Since X is the start of M, it can be observed that each occurrence of X is the sub-string of the prefix of M, which has a length equal to the length of X.
	\end{enumerate}
\end{obser}

The above observation forms the core of the first of the two stages in the algorithm.

\subsection{Stage 1: Finding maximum longest prefixes of each character in S}

The first stage splits the arbitrary string S into a list, K, of J sub-strings where J = Length of S. Where the first sub-string in the list is S.
Each of the sub strings in the K have the following properties
\begin{enumerate}
	\item Has a length $T = L - 1$, where L is the length of the sub-string prior to it in the list.
	\item Is a sub-string of S which starts at the character $C_{J - T} $ in S and ends at character $C_{J}$ in S.
\end{enumerate}

The stage then creates a list, U, which consist of J 0s. 
U has the following properties:
\begin{enumerate}
	\item Each element, $E_{i}$ is mapped to the corresponding element $C_{i}$ in S.
	\item Each element is a number $\ge 0$.
	\item Each element represents the maximum longest prefix which starts at $C_{i}$ in all of the sub-strings of K which has an element that maps to $C_{i}$.  
	
\end{enumerate}

Each sub-string, $K_{i}$, in K is then scanned iteratively for the longest prefix sub-string starting at each character, $K_{ij}$. If the value of the element in U which is mapped to by the character index of the $K_{ij}$ is less than the length of longest prefix at $K_{ij}$ then it is replaced by the newly found value. \\
  
In order to find the longest prefixes in the string S using a modified version of the O(n) longest prefix sub-string finder algorithm created by Michael G. Main and Richard J. Lorentz and documented in their research article \textit{An O(n log n) Algorithm for finding all repetitions in a String*} which was published in the Journal of Algorithms in 1984.\\

The pseudo code of the stage 1 algorithm follows:
\end{document}